\documentclass{article}
\usepackage{graphicx}
\usepackage{float}

\title{Process description for the creation of Unipath, a website for Ontario high school students going through the university admissions system}
\author{Soroush Paidar}
\date{Began July 2024}
\usepackage{amsmath}
\DeclareMathOperator{\errorfunction}{erf}
\begin{document}

    \maketitle

    \section{Introduction}
    I noticed that unlike in the United States, there are no tools for Ontario high school students to help them with the university admissions process, which is why I decided to build Unipath -- to help students like myself better understand the university admissions system.

    I wanted Unipath to have three main features, each of which will be described and outlined in their own sections:
    \begin{itemize}

        \item {Admissions chancing -- Using a student's credentials and comparing them to those of accepted and rejected students to produce a chance of acceptance}
        \item {Generating alternative options -- Using a student's credentials and their personal preferences for location, student population size, et cetera to suggest alternative universities and programs}
        \item {Help with supplemental applications -- Competitive programs such as Business Administration at the University of Western Ontario and Engineering at the University of Waterloo use supplemental applications to get a better sense of their applicants, often serving as a tiebreaker between students with similar academic achievements}

    \end{itemize}

    This document will outline the entirety of the process of creating Unipath from start to finish. For an overview of statistical concepts used in this project, see Introductory Statistics 2e. by Barbara Illowsky, Susan Dean, et al. and published by Openstax.

    \section{Features}
    \subsection{Admissions chancing}
    \subsubsection{Feature description}
    This is meant to be the main feature of Unipath. The user would input their coursework as well as any test scores and extracurricular activities for supplemental application purposes.
    \subsubsection{Process}
    Using data collected from applicants using polling and data released by universities themselves, I aimed to produce a normal distribution of acceptance averages for as many programs as possible considering the conditions for normal distribution. Originally, I attempted to do this by estimating changes in admissions odds in tandem with changes in grade-point average (GPA). I attempted to do this using CollegeVine, a service generating odds of admission to American universities. Using the odds of admission in comparison to GPA, I was able to model admission chance as a function of GPA\(^1\) for a number of sample schools, almost always being modeled as a cubic function. One example of those functions, modelling GPA vs admission probability for Boston College, was:

    \[p_{admit}(x) = 0.0005(9(x-0.8))^3 +5.2 + 1.2x\]

    Using these functions, I was able to estimate the change in change in admission probability. For admission to Boston College, this function was:

    \[p_{admit}'(x) = 1.0935(x-0.8)^2 + 1.2\]

    Ultimately, I decided that this method was both ineffective and that it could prove to be extremely inaccurate as CollegeVine considers many other factors in calculating admission probability due to the holistic nature of American university admissions. As a result, I decided to use existing data from Ontario universities and applicants to create chancing models. The majority of the data used for this was taken from the Common University Data Ontario system and a poll of admissions outcomes for applicants during the 2023-24 academic year.

    \textbf{Final method}

    My final plan was to use admissions data from CUDO to create normal distributions of grades. This would serve as exact data showing applicant's grades and using the provided applicant and acceptance counts, I can estimate median accepted averages. With that, I would be able to estimate grade ranges for acceptance.


    \textbf{Example Data for the Faculty of Arts and Science at the University of Toronto - St. George}


    \begin{table}[]
        \begin{center}
            \caption{Accepted grade ranges}
        \end{center}
        \begin{tabular}{lllllll}
            Program        & 95+     & 90-95   & 85-90   & 80-85  & 75-80  & 70-75  \\
            Soc. Sci.      & 28.70\% & 45.80\% & 20.30\% & 5.00\% & 0.30\% & 0\%    \\
            Phys/Math Sci. & 41.90\% & 44.10\% & 12.90\% & 1.10\% & 0\%    & 0\%    \\
            Humanities     & 27,70\% & 42.90\% & 21.60\% & 6.50\% & 1.10\% & 0.20\% \\
            Life Sciences  & 50.80\% & 39.60\% & 8.60\%  & 0.70\% & 0.30\% & 0\%    \\
            Rotman Comm.   & 64.40\% & 34.30\% & 1.20\%  & 0\%    & 0\%    & 0\%    \\
            Comp. Sci.     & 87.10\% & 12.90\% & 0\%     & 0\%    & 0\%    & 0\%
        \end{tabular}
    \end{table}

    \begin{table}[]
        \begin{center}
            \caption{Applicant count, standard deviation, and average "top 6"}
        \end{center}
        \begin{tabular}{llll}
            Program        & Applicants & Standard Dev. & Mean \\
            Soc. Sci.      & 7420       & 4.65          & 91.8 \\
            Phys/Math Sci. & 5211       & 4.112         & 93.4 \\
            Humanities     & 3567       & 5.1           & 91.5 \\
            Life Sciences  & 9892       & 4.08          & 94.1 \\
            Rotman Comm.   & 6734       & 3.248         & 95.2 \\
            Comp. Sci.     & 6204       & 3.05          & 96.9
        \end{tabular}
    \end{table}
    I used the data like that shown to produce normal distributions of accepted applicant's grades.

    When the user inputs their top 6 average, the program would generate a z-score considering their average and the mean and standard deviation for the chosen university program, and use that z-score to calculate a probability of admission.

    To calculate probability, one would usually use a Z-table to find the probability that corresponds to the Z-score computed. In order to keep the program efficient and accurate, it was decided that the best way to calculate probability is to calculate definite integrals of the distributions by finding the indefinite integral of the normal distribution.

    \[\int_{}^{}\frac{1}{\sigma\sqrt{2\pi}}e^{-\frac{1}{2}(\frac{x-\mu}{\sigma})^2}dx\]

    \[= \frac{1}{2}+\frac{1}{2}\errorfunction(\frac{x-\mu}{\sigma\sqrt{2}})\]

\end{document}
